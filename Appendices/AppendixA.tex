% Appendix A

\chapter{PNML grammar} % Main appendix title

\label{AppendixA} % For referencing this appendix elsewhere, use \ref{AppendixA}

\lhead{Appendix A. \emph{PNML grammar}} % This is for the header on each page - perhaps a shortened title

The official PNML grammar is available in www.pnml.org. It is written in
RELAX NG format. This are the main parts that I use in this work.

\section{RELAX NG implementation of PNML Core Model}

\begin{lstlisting}[label=grammar_core,caption=RELAX NG implementation of PNML Core Model]
<?xml version="1.0" encoding="UTF-8"?>

<grammar ns="http://www.pnml.org/version-2009/grammar/pnml"
  xmlns="http://relaxng.org/ns/structure/1.0"
  xmlns:a="http://relaxng.org/ns/compatibility/annotations/1.0"
  datatypeLibrary="http://www.w3.org/2001/XMLSchema-datatypes">

  <a:documentation>
    Petri Net Markup Language (PNML) schema.
    RELAX NG implementation of PNML Core Model.

    File name: pnmlcoremodel.rng
    Version: 2009
    (c) 2001-2009
    Michael Weber,
    Ekkart Kindler,
    Christian Stehno, 
    Lom Hillah (AFNOR)
    Revision:
    July 2008 - L.H
  </a:documentation>

  <include href="http://www.pnml.org/version-2009/grammar/anyElement.rng"/>

  <start>
    <ref name="pnml.element"/>
  </start>
  
  <define name="pnml.element">
    <element name="pnml">
      <a:documentation>
        A PNML document consists of one or more Petri nets.
        It has a version.
      </a:documentation>
      <oneOrMore>
        <ref name="pnml.content"/>
      </oneOrMore>
    </element>
  </define>

  <define name="pnml.content">
    <ref name="net.element"/>
  </define>

  <define name="net.element">
    <element name="net">
      <a:documentation>
        A net has a unique identifier (id) and refers to
        its Petri Net Type Definition (PNTD) (type).
      </a:documentation>
      <ref name="identifier.content"/>
      <ref name="nettype.uri"/>
      <a:documentation>
        The sub-elements of a net may occur in any order. 
        A net consists of at least a top-level page which 
        may contain several objects. A net may have a name,
        other labels (net.labels) and tool specific information in any order.   
      </a:documentation>
      <interleave>
        <optional>
          <ref name="Name"/>
        </optional>
        <ref name="net.labels"/>
        <oneOrMore>
          <ref name="page.content"/>
        </oneOrMore>
        <zeroOrMore>
          <ref name="toolspecific.element"/>
        </zeroOrMore>
      </interleave>
    </element>
  </define>

  <define name="identifier.content">
    <a:documentation>
      Identifier (id) declaration shared by all objects in any PNML model.
    </a:documentation>
    <attribute name="id">
      <data type="ID"/>
    </attribute>
  </define>

  <define name="nettype.uri">
    <a:documentation>
      The net type (nettype.uri) of a net should be redefined in the grammar
      for a new Petri net Type.
      An example of such a definition is in ptnet.pntd, the grammar
      for P/T Nets. The following value is a default.
    </a:documentation>
    <attribute name="type">
      <value>http://www.pnml.org/version-2009/grammar/pnmlcoremodel</value> 
    </attribute>
  </define>

  <define name="net.labels">
    <a:documentation>
      A net may have unspecified many labels. This pattern should be used
      within a PNTD to define the net labels.
    </a:documentation>
    <empty/>
  </define>

  <define name="basicobject.content">
    <a:documentation>
      Basic contents for any object of a PNML model.
    </a:documentation>
    <interleave>
      <optional>
        <ref name="Name"/>
      </optional>
      <zeroOrMore>
        <ref name="toolspecific.element"/>
      </zeroOrMore>
    </interleave>  
  </define> 


  <define name="page.content">
    <a:documentation>
      A page has an id. It may have a name and tool specific information.
      It may also have graphical information. It can also have many arbitrary labels.
      Note: according to this definition, a page may contain other pages. 
      All these sub-elements may occur in any order.
    </a:documentation>
    <element name="page">
      <ref name="identifier.content"/>  
      <interleave>
        <ref name="basicobject.content"/>
        <ref name="page.labels"/>
        <zeroOrMore>
          <ref name="netobject.content"/>
        </zeroOrMore>
        <optional>
          <element name="graphics">
            <ref name="pagegraphics.content"/>
          </element>
        </optional>
      </interleave> 
    </element>
  </define>

  <define name="netobject.content">
    <a:documentation>
      A net object is either a page, a node or an arc. 
      A node is a place or a transition, a reference place of
      a reference transition.
    </a:documentation>
    <choice>     
      <ref name="page.content"/>      
      <ref name="place.content"/>  
      <ref name="transition.content"/>    
      <ref name="refplace.content"/>
      <ref name="reftrans.content"/>
      <ref name="arc.content"/>
    </choice>
  </define>

  <define name="page.labels">
    <a:documentation>
      A page may have unspecified many labels. This pattern should be used
      within a PNTD to define new labels for the page concept.
    </a:documentation>
    <empty/>
  </define>

  <define name="place.content">
    <a:documentation>
      A place may have several labels (place.labels) and the same content 
      as a node.
    </a:documentation>
    <element name="place">
      <ref name="identifier.content"/>
      <interleave>
        <ref name="basicobject.content"/>
        <ref name="place.labels"/>
        <ref name="node.content"/>
      </interleave>
    </element>
  </define>

  <define name="place.labels">
    <a:documentation>
      A place may have arbitrary many labels. This pattern should be used
      within a PNTD to define the place labels.
    </a:documentation>
    <empty/>
  </define>

  <define name="transition.content">
    <a:documentation>
      A transition may have several labels (transition.labels) and the same 
      content as a node.
    </a:documentation>
    <element name="transition">
      <ref name="identifier.content"/>    
      <interleave>
        <ref name="basicobject.content"/>
        <ref name="transition.labels"/>
        <ref name="node.content"/>
      </interleave>
    </element>
  </define>

  <define name="transition.labels">
    <a:documentation>
      A transition may have arbitrary many labels. This pattern should be 
      used within a PNTD to define the transition labels.
    </a:documentation>
    <empty/>
  </define>

  <define name="node.content">
    <a:documentation>
      A node may have graphical information.
    </a:documentation>    
    <optional>
      <element name="graphics">
        <ref name="nodegraphics.content"/>
      </element>
    </optional>  
  </define>

  <define name="reference">
    <a:documentation>
      Here, we define the attribute ref including its data type.
      Modular PNML will extend this definition in order to change
      the behavior of references to export nodes of module instances.
    </a:documentation>
    <attribute name="ref">
      <data type="IDREF"/>
    </attribute>
  </define>

  <define name="refplace.content">
    <a:documentation>
      A reference place is a reference node.
    </a:documentation>
    <a:documentation>
      Validating instruction: 
      - _ref_ MUST refer to _id_ of a reference place or of a place.
      - _ref_ MUST NOT refer to _id_ of its reference place element.
      - _ref_ MUST NOT refer to a cycle of reference places.
    </a:documentation>
    <element name="referencePlace">
      <ref name="refnode.content"/>
    </element>
  </define>

  <define name="reftrans.content">
    <a:documentation>
      A reference transition is a reference node.
    </a:documentation>
    <a:documentation>
      Validating instruction: 
      - The reference (ref) MUST refer to a reference transition or to a 
      transition.
      - The reference (ref) MUST NOT refer to the identifier (id) of its
      reference transition element.
      - The reference (ref) MUST NOT refer to a cycle of reference transitions.
    </a:documentation>
    <element name="referenceTransition">
      <ref name="refnode.content"/>
    </element>
  </define>

  <define name="refnode.content">
    <a:documentation>
      A reference node has the same content as a node.
      It adds a reference (ref) to a (reference) node.
    </a:documentation>
    <ref name="identifier.content"/>
    <ref name="reference"/>
    <ref name="basicobject.content"/>
    <ref name="node.content"/>  
  </define>

  <define name="arc.content">
    <a:documentation>
      An arc has a unique identifier (id) and
      refers both to the node's id of its source and 
      the node's id of its target.
      In general, if the source attribute refers to a place, 
      then the target attribute refers to a transition and vice versa.
    </a:documentation>
    <element name="arc">
      <ref name="identifier.content"/>
      <attribute name="source">
        <data type="IDREF"/>
      </attribute>
      <attribute name="target">
        <data type="IDREF"/>
      </attribute>
      <a:documentation>
        The sub-elements of an arc may occur in any order.
        An arc may have a name, graphical and tool specific information.
        It may also have several labels.
      </a:documentation>
      <interleave>
        <optional>
          <ref name="Name"/>
        </optional>
        <ref name="arc.labels"/>
        <optional>
          <element name="graphics">
            <ref name="edgegraphics.content"/>
          </element>
        </optional>
        <zeroOrMore>
          <ref name="toolspecific.element"/>
        </zeroOrMore>
      </interleave>
    </element>
  </define>

  <define name="arc.labels">
    <a:documentation>
      An arc may have arbitrary many labels. This pattern should be used
      within a PNTD to define the arc labels.
    </a:documentation>
    <empty/>
  </define>

  <define name="pagegraphics.content">
    <a:documentation>
      A page graphics is actually a node graphics
    </a:documentation>
    <ref name="nodegraphics.content"/> 
  </define>

  <define name="nodegraphics.content">
    <a:documentation>
      The sub-elements of a node's graphical part occur in any order.
      At least, there may be one position element.
      Furthermore, there may be a dimension, a fill, and a line element.
    </a:documentation>
    <interleave>
      <ref name="position.element"/>
      <optional>
        <ref name="dimension.element"/>
      </optional>
      <optional>
        <ref name="fill.element"/>
      </optional>
      <optional>
        <ref name="line.element"/>
      </optional>
    </interleave>
  </define>

  <define name="edgegraphics.content">
    <a:documentation>
      The sub-elements of an arc's graphical part occur in any order.
      There may be zero or more position elements.
      Furthermore, there may be a line element.
    </a:documentation>
    <interleave>
      <zeroOrMore>
        <ref name="position.element"/>
      </zeroOrMore>
      <optional>
        <ref name="line.element"/>
      </optional>
    </interleave>
  </define>

  <define name="simpletext.content">
    <a:documentation>
      This definition describes the contents of simple text labels
      without graphics.
    </a:documentation>
    <optional>
      <element name="text">
        <a:documentation>
          A text should have a value. 
          If not, then there must be a default.
        </a:documentation>
        <text/>
      </element>
    </optional>
  </define>

  <define name="annotationstandard.content">
    <a:documentation>
      The definition annotationstandard.content describes the 
      standard contents of an annotation.
      Each annotation may have graphical or tool specific information.
    </a:documentation>
    <interleave>
      <optional>
        <element name="graphics">
          <ref name="annotationgraphics.content"/>
        </element>
      </optional>
      <zeroOrMore>
        <ref name="toolspecific.element"/>
      </zeroOrMore>
    </interleave>
  </define>

  <define name="simpletextlabel.content">
    <a:documentation>
      A simple text label is an annotation to a net object containing 
      arbitrary text.
      Its sub-elements occur in any order.
      A simple text label behaves like an attribute to a net object.
      Furthermore, it contains the standard annotation contents which 
      basically defines the graphics of the text.
    </a:documentation>
    <interleave>
      <ref name="simpletext.content"/>
      <ref name="annotationstandard.content"/>
    </interleave>
  </define>

  <define name="Name">
    <a:documentation>
      Label definition for a user given name of an 
      element.
    </a:documentation>
    <element name="name">
      <ref name="simpletextlabel.content"/>
    </element>
  </define>

  <define name="annotationgraphics.content">
    <a:documentation>
      An annotation's graphics part requires an offset element describing
      the offset the center point of the surrounding text box has to
      the reference point of the net object on which the annotation occurs.
      Furthermore, an annotation's graphic element may have a fill, a line, 
      and font element.
    </a:documentation>
    <ref name="offset.element"/>
    <interleave>
      <optional>
        <ref name="fill.element"/>
      </optional>
      <optional>
        <ref name="line.element"/>
      </optional>
      <optional>
        <ref name="font.element"/>
      </optional>
    </interleave>
  </define>

  <define name="position.element">
    <a:documentation>
      A position element describes Cartesian coordinates.
    </a:documentation>
    <element name="position">
      <ref name="coordinate.attributes"/>
    </element>
  </define>

  <define name="offset.element">
    <a:documentation>
      An offset element describes Cartesian coordinates.
    </a:documentation>
    <element name="offset">
      <ref name="coordinate.attributes"/>
    </element>
  </define>

  <define name="coordinate.attributes">
    <a:documentation>
      The coordinates are decimal numbers and refer to an appropriate 
      xy-system where the x-axis runs from left to right and the y-axis
      from top to bottom.
    </a:documentation>
    <attribute name="x">
      <data type="decimal"/>
    </attribute>
    <attribute name="y">
      <data type="decimal"/>
    </attribute>
  </define>

  <define name="dimension.element">
    <a:documentation>
      A dimension element describes the width (x coordinate) and height 
      (y coordinate) of a node.  
      The coordinates are actually positive decimals.
    </a:documentation>
    <element name="dimension">
      <attribute name="x">
        <ref name="positiveDecimal.content"/>
      </attribute>
      <attribute name="y">
        <ref name="positiveDecimal.content"/>
      </attribute>
    </element>
  </define>

  <define name="positiveDecimal.content" ns="http://www.w3.org/2001/XMLSchema-datatypes">
    <a:documentation>
      Definition of a restricted positive decimals domain with a total digits 
      number of 4 and 1 fraction digit. Ranges from 0 to 999.9
    </a:documentation>
    <data type='decimal'>
      <param name='totalDigits'>4</param>
      <param name='fractionDigits'>1</param>
      <param name='minExclusive'>0</param>
    </data>
  </define>

  <define name="fill.element">
    <a:documentation>
      A fill element describes the interior colour, the gradient colour,
      and the gradient rotation between the colors of an object.  If an
      image is available the other attributes are ignored.
    </a:documentation>
    <element name="fill">
      <optional>
        <attribute name="color">
          <ref name="color.type"/>
        </attribute>
      </optional>
      <optional>
        <attribute name="gradient-color">
          <ref name="color.type"/>
        </attribute>
      </optional>
      <optional>
        <attribute name="gradient-rotation">
          <choice>
            <value>vertical</value>
            <value>horizontal</value>
            <value>diagonal</value>
          </choice>
        </attribute>
      </optional>
      <optional>
        <attribute name="image">
          <data type="anyURI"/>
        </attribute>
      </optional>
    </element>
  </define>

  <define name="line.element">
    <a:documentation>
      A line element describes the shape, the colour, the width, and the 
      style of an object.
    </a:documentation>
    <element name="line">
      <optional>
        <attribute name="shape">
          <choice>
            <value>line</value>
            <value>curve</value>
          </choice>
        </attribute>
      </optional>
      <optional>
        <attribute name="color">
          <ref name="color.type"/>
        </attribute>
      </optional>
      <optional>
        <attribute name="width">
          <ref name="positiveDecimal.content"/>
        </attribute>
      </optional>
      <optional>
        <attribute name="style">
          <choice>
            <value>solid</value>
            <value>dash</value>
            <value>dot</value>
          </choice>
        </attribute>
      </optional>
    </element>
  </define>

  <define name="color.type">
    <a:documentation>
      This describes the type of a color attribute. Actually, this comes
      from the CSS2 (and latest versions) data type system.
    </a:documentation>
    <text/>
  </define>

  <define name="font.element">
    <a:documentation>
      A font element describes several font attributes, the decoration, 
      the alignment, and the rotation angle of an annotation's text.
      The font attributes (family, style, weight, size) should be conform
      to the CSS2 and latest versions data type system.
    </a:documentation>
    <element name="font">
      <optional>
        <attribute name="family">
          <text/>  <!-- actually, CSS2 and latest versions font-family -->
        </attribute>
      </optional>
      <optional>
        <attribute name="style">
          <text/>  <!-- actually, CSS2 and latest versions font-style -->
        </attribute>
      </optional>
      <optional>
        <attribute name="weight">
          <text/>  <!-- actually, CSS2 and latest versions font-weight -->
        </attribute>
      </optional>
      <optional>
        <attribute name="size">
          <text/>  <!-- actually, CSS2 and latest versions font-size -->
        </attribute>
      </optional>
      <optional>
        <attribute name="decoration">
          <choice>
            <value>underline</value>
            <value>overline</value>
            <value>line-through</value>
          </choice>
        </attribute>
      </optional>
      <optional>
        <attribute name="align">
          <choice>
            <value>left</value>
            <value>center</value>
            <value>right</value>
          </choice>
        </attribute>
      </optional>
      <optional>
        <attribute name="rotation">
          <data type="decimal"/>
        </attribute>
      </optional>
    </element>
  </define>

  <define name="toolspecific.element">
    <a:documentation>
      The tool specific information refers to a tool and its version.
      The further substructure is up to the tool.
    </a:documentation>
    <element name="toolspecific">
      <attribute name="tool">
        <text/>
      </attribute>
      <attribute name="version">
        <text/>
      </attribute>
      <zeroOrMore>
        <ref name="anyElement"/>
      </zeroOrMore>
    </element>
  </define>
  
</grammar>
\end{lstlisting}







\newpage

\section{RELAX NG implementation of Petri Net Type Definition for Place/Transition nets}

\begin{lstlisting}[label=grammar_place_transitions,caption=RELAX NG implementation of PNTD for Place/Transition nets]
<?xml version="1.0" encoding="UTF-8"?>

<grammar ns="http://www.pnml.org/version-2009/grammar/pnml"
  xmlns="http://relaxng.org/ns/structure/1.0"
  xmlns:a="http://relaxng.org/ns/compatibility/annotations/1.0"> 

  <a:documentation>
    RELAX NG implementation of Petri Net Type Definition for Place/Transition nets.
    This PNTD re-defines the value of nettype.uri for P/T nets.
    
    File name: ptnet.pntd
    Version: 2009
    (c) 2007-2009
    Lom Hillah (AFNOR)
    Revision:
    July 2008 - L.H
  </a:documentation>

  <a:documentation>
    The PT Net type definition.
    This document also declares its namespace.
    All labels of this Petri net type come from the Conventions document.
    The use of token graphics as tool specific feature is possible.
  </a:documentation>

  <include href="http://www.pnml.org/version-2009/grammar/conventions.rng"/>
  
  <!--
  <include href="http://www.pnml.org/version-2009/grammar/pnmlextensions.rng"/>
    We do not need to include this, because the pnmlcoremodel.rng covers any
    toolspecific extension.
  -->
  
  <include href="http://www.pnml.org/version-2009/grammar/pnmlcoremodel.rng"/>
  
  <define name="nettype.uri" combine="choice">
    <a:documentation>
      The URI value for the net type attribute, 
      declaring the type of P/T nets.
    </a:documentation>
    <attribute name="type">    
      <value>http://www.pnml.org/version-2009/grammar/ptnet</value>   
    </attribute>
  </define>

  
  <define name="PTMarking">
    <a:documentation>
      Label definition for initial marking in nets like P/T-nets.
      <contributed>Michael Weber</contributed>
      <date>2003-06-16</date>
      <reference>
        W. Reisig: Place/transition systems. In: LNCS 254. 1987.
      </reference>
    </a:documentation>
    <element name="initialMarking">
      <ref name="nonnegativeintegerlabel.content"/>
    </element>
  </define>

  <define name="PTArcAnnotation">
    <a:documentation>
      Label definition for arc inscriptions in P/T-nets.
      <contributed>Michael Weber, AFNOR</contributed>
      <date>2003-06-16</date>
      <reference>
        W. Reisig: Place/transition systems. In: LNCS 254. 1987.
      </reference>
    </a:documentation>
    <element name="inscription">
      <ref name="positiveintegerlabel.content"/>
    </element>
  </define>

  <define name="place.labels" combine="interleave">
    <a:documentation>
      A place of a P/T net may have an initial marking.
    </a:documentation> 
    <optional><ref name="PTMarking"/></optional>
  </define>

  <define name="arc.labels" combine="interleave">
    <a:documentation>
      An arc of a P/T net may have an inscription.
    </a:documentation>
    <optional><ref name="PTArcAnnotation"/></optional>
  </define>

</grammar>
\end{lstlisting}
