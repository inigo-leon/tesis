% Chapter 1: Introduction

\chapter{Introduction} % Main chapter title

\label{Chapter1} % Change X to a consecutive number; for referencing this chapter elsewhere, use \ref{Chapter1}

\lhead{Chapter 1. \emph{Introduction}}


\section{Background of the research}
Petri nets are widespread for modeling many
classes of systems, such as manufacturing logistics processes and services
\citep{SM-Jimenez2004143,SM-Guasch2002}, concurrent systems \citep{EPN-Jensen2009}. However, all these nets are described in a comprehensive way and must have the information of the entire net to determine
its evolution.
Furthermore, these nets can be modified with no control of integrity or authoring,
for example. 
\section{Research problem}
The problem occurs when somebody doesn't want to describe the whole subnet.
Or, maybe, is wanted that one part of the process is only accessible for one
specific person or entity.


The first approach to solve this problem is to
take two Petri nets


\begin{itemize}
\item
One Petri Net with the public information, extracting the private data. This
is an incomplete model of the process
\item
Another Petri Net with the whole information
for the interested person or entity.

\end{itemize}  

As you can notice, this is not an efficient way to publish this kind of Petri
Nets.

Other problem appears when I want to protect parts of the net from undesired
modifications or ensure the authoring of some parts (or the whole net).


\section{Justification of the research}
It would be interesting to provide security to a Petri net:


\begin{itemize}
\item 
hiding a part of it. This can be useful, for example,
distributing a process we want to be secret \citep{HID-Inigo2011MT},
or simply to be a part of the net to be complex and do not
interested handle for any reason \citep{HID-Inigo2011MT}.

\item avoiding not allowed changes in it (or a part of it).
\item authenticating it (or a part of it). Useful to ensure who has developed
a Petri net or subnet.
\item avoiding the possibility of supplant other people in the authority of
the Petri net or some of its parts.

\end{itemize}
So here is my contribution. I have researched the possibilities of hiding
a part of a Petri Net so that everybody can access the public information,
maintaining the secret of the private data. This private data is accessible only
for authorized people. And not only that: I ensure data integrity, authentication
and non repudiation to Petri nets or subnets.

Some authors study the possibilities of Petri nets reduction \citep{SN-Valette197935,SN-Suzuki198351,SN-Fahmy1990321,SN-DRUZHININVA19921922,SN-Fahmy1993127,R-Xia20111662}, grouping
in one place or transition a subnet, so that what happens
on this subnet, is encapsulated in a single point of execution. However,
we want to go further by defining parts of the net that are hidden (not clustered) and what are the implications, studied within
network properties.

The main objective of this thesis is to take parts of a Petri nets and provide
them of wide security (privacy, integrity, authentication and non repudiation).



\section{Methodology}


In order to achieve this goal, I have defined three milestones:

\begin{enumerate}
\item
Extend Petri Nets definition in order to define subnets, abstracting the
internal structure from the the rest of the net, focussing on hiding information. 
\item
Choose a lossless and extendible representation
of this kind of Petri Nets
\item
Define a hiding and signing method for this representation
\end{enumerate}



For the first milestone, I work for the creation
of the theoretical basis for further study of Petri nets
in which certain parts are hidden. So we setup
a generic framework of definitions and notations that allow us to deepen
in the study of the characteristics and properties of Petri nets and their
subnets \citep{G-Murata1989541,G-Silva1985}.
Also mention work already carried
by other researchers in which we rely for
our goal (i.e. \citep{SM-Silva19931,EPN-David2010,EPN-Jensen2009,G-EPN-Peterson1981}). All this will be necessary to create the framework that allows us to study occultation in Petri nets. We will expand the vision of Petri nets, providing them with greater functionality,
such us attachable subnet.

The next step in this work is to choose (or define) a flexible representation of Petri
nets that allows us to translate the previous extended nets. This representation has to be really extendible and flexible in order to be able to show actual and future characteristics of Petri nets.
I can advance you that the selected representation is the standard PNML and I have to define and extension for it in order to represent subnets that are going to
be hidden.

Once selected this representation, the last step is the hiding and signing method. Once
more, I bet for standard protocols like XMLEncription and XMLSignature.

This is a very basic investigation because I extend the very early
definitions of Petri nets. Because of it, the results of this thesis is very probably extensible to any other development whose base are the classic Petri Nets. For example, I am not going to study colored Petri nets, neither timed
Petri nets, etc. But it is very easy to see that the results achieved in
this thesis can be applied to them with no problem.  

\section{Delimitations of scope and key assumptions}  
For this work we will always deal with ordinary and pure networks, unless otherwise expressly. This assumption is only for clarity reasons, because
the protocols and methods described in this work are perfectly extensible
to other kind of Petri nets, as long as these Petri nets are representable
in PNML format.





