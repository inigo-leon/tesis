% Chapter Template

\chapter{PNML} % Main chapter title

\label{Chapter3} % Change X to a consecutive number; for referencing this chapter elsewhere, use \ref{Chapter3}

\lhead{Chapter 3. \emph{Petri Net Marked Language (PNML)}} % Change X to a consecutive number; this is for the header on each page - perhaps a shortened title

%----------------------------------------------------------------------------------------
%       SECTION 1
%----------------------------------------------------------------------------------------
\section{Introduction}
Petri Net Marked Language (PNML) is an XML language created for represent Petri Nets. With this language we can take a Petri Net and store it into an XML file without loss of information.
But PNML hasn't got a way to represent subnets. So I am going to extend PNML language in order to get several goals:
\begin{enumerate}
\item Represent subnets of a Petri Net [3].
\item Include input and output interfaces for every subnet.
\end{itemize}

%----------------------------------------------------------------------------------------
%       SECTION 2
%----------------------------------------------------------------------------------------
\section{PNML grammar}

%----------------------------------------------------------------------------------------
%       SECTION 3
%----------------------------------------------------------------------------------------
\section{PNML examples}

%----------------------------------------------------------------------------------------
%       SECTION 4
%----------------------------------------------------------------------------------------
\section{PNML extension for representing subnets}

If we take the PNML grammar, there are lots of tags, but we are going to take some of them, without loss of generality. The official grammar is described and can be downloaded from  the official page of PNML (www.pnml.org).
Inside a PNML document there are three main elements. This elements have required tags:

\begin{itemize}
 \item place: define a place in the petri net with an id and a name. This
 corresponds with a column in the incidence matrix
\begin{verbatim}
        <place id="p1">
                <name>
                <text>Place one</text>
                </name>
        </place>
\end{verbatim}
\end{itemize}

----------


