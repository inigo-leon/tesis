% Chapter Template

\chapter{Literature review} % Main chapter title

\label{LiteratureReview} % Change X to a consecutive number; for referencing this chapter elsewhere, use \ref{ChapterX}

\lhead{Chapter 2. \emph{Literature review}} % Change X to a consecutive number; this is for the header on each page - perhaps a shortened title

//BORRAR

\begin{itemize}
 \item G - Petri nets general
 \item SM - Simulacion de modelos
 \item EPN - Extensiones de Petri nets originales
 \item R - Reduccion de redes
 \item SN - Subredes
 \item REP - Representation
 \item PROP - Propiedades de PN
 \item HID - Ocultacion de partes
 \item PNML - PNML
 \item PNMLE - Extensiones PNML

\end{itemize}

//

In this chapter I am going to go over the ancient Petri net history. 
This work is a very basic investigation 
on Petri nets. This means that the most of the references are very general.

For this literature review, I will follow the structure of this thesis:

\begin{enumerate}
\item First of all, I am going to describe some generalities of Petri nets
as an introduction\item Then I will describe subnets and the process of splitting Petri nets
into several subnets. Some of those subnets are going to be hidden 
\item The next step is to explain some possible Petri net representations
and my selection of PNML for the hiding process
\item Once selected PNML, I am going to extend this language to support the
subnets defined before
\item The last step is the hiding properly speaking. To do this, I will use the standard XMLEncryption for ciphering the secret information
\end{enumerate}

\section{Introduction. Petri nets}

In the 60's, Carl Adam Petri invented a new way to describe distributed systems
called Petri nets \cite{G-Petri1962PhD,G-Petri1966,G-Petri1976}. Many net
theories
are based on those works\cite{G-Petri2007}. Nowadays,
Petri nets are really extended to represent discrete systems \cite{SM-Holloway1997151,EPN-SM-Latorre2010152,EPN-SM-Latorre2010247,EPN-SM-Silva2011427}. There are lots of applications
of Petri nets:

\begin{itemize}
\item Modelling of sequential processes\cite{SM-Recalde1998267}, concurrent
systems\cite{EPN-SM-Jensen2007213,EPN-SM-Kristensen200819}, manufacturing \cite{G-Silva1989374,SM-Desrochers2010,SM-Silva19931,SM-Silva1997182,G-Silva1989374}, logistic
processes \cite{SM-Guasch2002},...
\item Simulation of industrial applications \cite{SM-Jimenez2006159,SM-Latorre2013346}, logistic and production systems \cite{SM-Jimenez2004143},...
\end{itemize} 

//TODO Incluir mas aplicaciones de las redes de Petri

This work is not about of Petri net application, so I am not
going to deepen this field. 
However I really mind the intrinsic structure of them. Since the definition
of Petri nets,
many authors investigated about them. The basic basis of my work are some of the best Petri net researchers, who are included in this review:
\begin{itemize}
\item Murata, T \cite{G-Murata1977412,G-SM-Murata19772,G-Murata1989541}
\item Silva, M \cite{G-Silva1985,G-Silva1993,G-Silva201213}
\item Peterson, JL \cite{G-EPN-Peterson1981}
\end{itemize} 

And not only general Petri nets. Any kind of extensions are well received
\begin{itemize}
\item Jensen, K \cite{EPN-SM-Jensen2007213,G-EPN-Jensen1985723}
\item ...

\end{itemize}

\section{Subnets}
\section{Petri nets representation. PNML}
\section{PNML extension for hiding support}
\section{Hide information on Petri nets. XMLEncryption}




