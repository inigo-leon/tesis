% Chapter 1: Introduction

\chapter{Introduction} % Main chapter title

\label{Chapter1} % Change X to a consecutive number; for referencing this chapter elsewhere, use \ref{Chapter1}

\lhead{Chapter 1. \emph{Introduction}}

Petri nets are widespread for modeling many
classes of systems, such as manufacturing logistics processes and services
\cite{guasch} \cite{jimenez}, concurrent systems \cite{jensen}. However, all these nets are described in a comprehensive way and must have the information of the entire net to determine
its evolution. It would be interesting to take a Petri net and
hide a part of it. This can be useful, for example,
distributing a process we want is some secret \cite{inigo},
or simply to be a part of the net to be complex and do not
interested handle for any reason \cite{jimenez}. In advanced work, we study the possibilities of Petri nets reduction \cite{xia}, grouping
in one place or transition a subnet, so that what happens
on this subnet, is encapsulated in a single point of execution. However,
we want to go further by defining parts of the net that are hidden (not clustered) and what are the implications, studied within
network properties. The aim of this work is the creation
of the theoretical basis for further study of Petri nets
in which certain parts are hidden. So we setup
a generic framework of definitions and notations that allow us to deepen
in the study of the characteristics and properties of Petri nets.
We will expand the vision of Petri nets, providing them with greater functionality.

The first part of this work, we study the state
the art in this field. We are going to deepen in the 
basic Petri nets basic definitions and properties \cite{murata}. Also mention work already carried
by other researchers in which we rely for
our goal. All this will be necessary to create the framework that allows us to study occultation in Petri nets.

Then we will go on to the description of the process of hiding a subnet

\begin{nota}
For this work we will always deal with ordinary and pure networks, unless otherwise expressly.
\end{nota}
