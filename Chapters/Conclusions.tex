% Chapter Template

\chapter{Conclusions} % Main chapter title

\label{Chapter: Conclusions} % Change X to a consecutive number; for referencing this chapter elsewhere, use \ref{Chapter4}

\lhead{Chapter . \emph{Conclusions}} % Change X to a consecutive number; this is for the header on each page - perhaps a shortened title

\textbf{Intentar que sea un resumen de todo lo tratado, haciendo hincapie en lo aportado
y en futuras lineas de investigacion. Que no sea corta. Lo suficientemente
larga para que alguien que se lea la tesis pueda recordarla completamente}

Throughout this paper we have presented Petri nets with definitions and basic properties. From this initial presentation, have been building a series of elements as a basis for further investigation. We defined subnets, subnets classifications have been studied, we have defined front-ends (interfaces) for those subnets, etc.. From this point is possible a further study of these subnets (their properties, utilities,....)

The contribution of this research is to establish the basis for the methodological study of hiding parts of Petri nets. We present the definitions and some basic properties and from here is to complete it by preparing a doctoral thesis.
