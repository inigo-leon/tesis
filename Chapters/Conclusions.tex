% Chapter Template

\chapter{Conclusions} % Main chapter title

\label{Chapter: Conclusions} % Change X to a consecutive number; for referencing this chapter elsewhere, use \ref{Chapter4}

\lhead{Chapter . \emph{Conclusions}} % Change X to a consecutive number; this is for the header on each page - perhaps a shortened title

\textbf{Intentar que sea un resumen de todo lo tratado, haciendo hincapie en lo aportado
y en futuras lineas de investigacion. Que no sea corta. Lo suficientemente
larga para que alguien que se lea la tesis pueda recordarla completamente}

Throughout this paper I have enriched Petri nets with definitions and properties. From this initial presentation, have been building a series of elements as a basis for further investigation. We defined subnets, subnets classifications have been studied, we have defined front-ends (interfaces) for those subnets, etc.. From this point is possible a further study of these subnets (their properties, utilities,....)
and the methodological study of securing parts of Petri nets. 

In the first chapter of this thesis I have introduced the research problem.
The nets are represented in a comprehensive way, so that the whole information is visible to everybody. Furthermore, these nets are not prepared to avoid undesired changes or to ensure the authoring of them.

So here is my contribution  to the knowledge: to provide security to a Petri net. The aspects covered by this investigation are:


\begin{itemize}
\item 
to occult a part of the net (or entire). The secret is maintained, and all the information
is stored in the same file, but hidden.
This information is only available to accredited people.
\item to avoid unwanted changes.
Any modification is, at least, detected.\item to authenticate the net (or a part of it). We know who has developed
a Petri net or subnet.
\item to avoid the possibility of supplant other people in the authority of
the Petri net or some of its parts.
\end{itemize}


The next chapter is the state of the art. In this chapter, the literature
about subnets, hiding, encryption, Petri net representation, PNML extensions,
Petri net securing, etc. are grouped and analyzed in order to understand the general knowledge about these topics, related to my objectives. The general conclusions in this part are:
\begin{itemize}
\item There are lots of..... 
\end{itemize}
 
   
